\chapter{Wildland fire behavior} 



	\citet{rothermel1983} derives the value for fireline intensity from two measurable variables: the amount of energy available for release within the unburned fuel (heat per unit area), and how fast the flame front moves through the fuel (rate of spread).
	\citet{byram1959} provides an equation to calculate intensity $I$ after the fact\textemdash $I = H \cdot w \cdot r$ \textemdash if one collects the relevant data: $H$ is heat yield, obtainable by putting fuel clipped before the fire in a bomb calorimeter; $w$ is the amount of fuel consumed, determined by subtracting post-burn fuel from pre-burn fuel load measurements; and $r$ is the rate of spread.
	\index{Byram's fireline intensity equation} 
	
		It is important to note that because wind tends to push flames down, as illustrated in Fig.~\ref{fig:AussieHeadFire}, the flame parameter best associated with fireline intensity is true \emph{flame length}\textemdash the distance from the base of the flame in the fuelbed, to its tip\textemdash and not just \emph{flame height}, which does not increase in proportion to wind speed precisely because the wind lays the flame down. 
		The non-linear relationship between intensity and flame height complicates post-hoc measurements of fire behaviour such as scorch height on trees.
		
Weather, fuel, and topography all interact to determine the outcomes of fire spread \citep{holsinger2016}. 
