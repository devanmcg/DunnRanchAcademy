\chapter{Fire management basics}

As wind speed increases, so does the rate of spread in the same direction. 
It is essential to know the speed and intensity of a head fire when making tactical decisions.
While lower-intensity head fires can be attacked directly\textemdash by specialised crews with hand tools and water hoses\textemdash fast-moving flame fronts with long flames must be attacked indirectly\textemdash by creating fire breaks and removing fuel between the barrier and the oncoming flame front.

\section{Situational Awareness}

\section{LCES}

\section{Watch-out situations}

\section{Ten Standard Orders} 

\section{Common denominators} 

\section{Driving safety} 
\textbf{A R R I V E~~~A L I V E} 
Always drive defensively.
Reducing response vehicle speed
can prevent rollovers.
Red traffic signals and stop signs
mean complete STOP.
Insist that vehicle occupants use
seat belts.
Verify vehicle occupants are
seated and belted.
Evaluate road surface and
weather conditions.
Abide by federal and state motor
vehicle laws.
Lengthy response distances
require frequent rest stops.
Initiate standard vehicle backing operating procedures.
Value occupant and public safety over time and speed.
Enter dangerous curves and intersections cautiously.