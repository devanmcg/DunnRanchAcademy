\section{About the author} 

\subsection*{Devan Allen McGranahan}

\begin{itemize}
	\item[$\bullet$] Education 
	\begin{itemize}
		\item PhD, Ecology \& Evolutionary Biology, Iowa State University (2011)
		\item MSc, Sustainable Agriculture and Animal Ecology, Iowa State University (2008)
		\item BA, Biology, Grinnell College (2004)
	\end{itemize}
	\item[$\bullet$] Employment history 
	\begin{itemize}
		\item Assistant Professor of Range Science, North Dakota State University, Fargo, ND (2014-present)
		\item Fulbright visiting scholar, Department of Grassland Science, University of KwaZulu-Natal, Pietermaritzburg, South Africa (2013-2014)
		\item Mellon Post-doctoral Fellow, Environmental Studies, The University of the South, Sewanee, TN (2011-2013)
	\end{itemize}
	\item[$\bullet$] Relevant experience 
	\begin{itemize} 
		\item 40+ contributions to the scientific literature
		\item Developer of wildland fire science curriculum and teaching materials; instructor of undergraduate and graduate-level wildland fire science courses
		\item 10 years experience researching rangeland fire behavior and fire effects 
		\item 20 years experience conducting prescribed fire in grasslands
		\item 4 years experience leading prescribed fire operations in North Dakota grasslands
		\item Wildland Firefighter FFT2 certified; substantial National Wildfire Coordinating Group training in wildland fuels, fire behavior, fire effects, ignition operations, and leadership. 
	\end{itemize}	
\end{itemize}

I grew up working for my father on our family owned-and-operated farm in Clay County, Iowa, which included a cow-calf operation managed with a multi-paddock rotational grazing system. 
As such I am familiar with the challenges of making a living from natural resources subject to environmental factors both helpful and harmful. 

As a researcher and educator in the land grant university system, I seek to create and disseminate the best knowledge available to enhance the profitability and sustainability of farm and ranch livelihoods. 
I am committed to supporting stakeholders and partners with the best science available, and promoting fact-based understanding of fire and grazing in the public sphere. 