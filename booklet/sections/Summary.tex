\begin{itemize} 
	\item[$\bullet$] On 3 April 2013, the Pasture 3B prescribed fire on the Grand River National Grasslands in Perkins County, SD escaped and spread across Federal and private land. 
	After emergency managers misread the name, the incident became known as the Pautre Fire. 

	\item[$\bullet$] Several landowners have filed suit against the US Forest Service, which manages the National Grasslands and conducted the Pasture 3B prescribed fire. 
	In this report, I consider several claims for damages due to lost forage resources, as well as claims for damages due to other losses including fencing and treed shelterbelts. 
	In considering these claims, I rely on an extensive body of peer-reviewed scientific literature on rangeland fire effects from the US Great Plains. 
	
	\item[$\bullet$] This report summarizes my opinions on the reasonableness of plaintiffs' claims: 
	
	\begin{itemize}
		\item I found no support for damages caused to forage or grazing. 
		Rangelands of the Northern Great Plains are resilient to fire. 
		The scientific literature, including data collected from the Pautre Fire, indicates that (a) fire effects on native prairie productivity returns to\textemdash or exceeds\textemdash that of pre-fire or unburned rangeland by the season after fire, and (b) it is unnecessary to defer grazing in the season immediately after fire.
		Plaintiffs have provided no evidence to substantiate their claims that rangeland on their ranches demonstrated a different response than observed in these studies.
		
		\item Metal fencing materials are resistant to grassland fire. 
		Loss of integrity of these components is mostly attributable to age, not fire. 
		Thus, in the event damages are awarded for the repair of fencelines and replacement of wooden posts, the award should discount from the full value an amount proportional to the age of the fence. 
		
		\item Many trees used in shelterbelts are short-lived. 
		Damages awarded for shelterbelts should be discounted in proportion to the remaining life of trees were there no fire.
		
		\item Finally, although the intent of the pasture-to-prairie reconstruction quotation is not clear, the plan far exceeds recommended management practices for weed-invaded rangeland. 
		Weed abundance following fire is mostly attributable to failure to control weed populations prior to being burned. 
		Thus, I found no support for awarding damages related to post-fire weed problems.    
	\end{itemize}
\end{itemize}